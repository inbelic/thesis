\documentclass{article}

\usepackage{amsmath, amssymb}

\begin{document}
\noindent
\begin{center}
%Title
{\Large{\bf Quantifying Growth Modelling of Computational Models of Angiogenesis\\}}
\vspace{0.2 cm}
{\large Christopher Finn Plummer} \\
{\small Supervisor: Gesina Menz}
\end{center}
\vskip 1 cm

\section{Background}
The vascular network is the network of different types of blood vessels such as
arteries, capillaries and veins through which blood flows through our bodies.
How these blood vessels grow plays an important role in processes such as
embryonal as well as tumour development as blood needs to reach
every part of the tissue to deliver oxygen and discard waste products.

Specifically for tumour development, angiogenesis describes the formation of new
blood vessels from already existing ones. Once these new blood vessels reach 
the tumour, this starts a critical new phase of tumour development which we want
to understand better. To promote this understanding,
different mathematical and computational models exist that simulate the
angiogenic process, however, it is difficult to quantify how well the vascular
networks generated by those models function compared to the networks found in
vivo or in vitro experiments.

Therefore, we are motiviated to develop a technique or process to compare
such models. To do so we will need to initially have a way to represent the
networks as graphs. Given the graph representations of two networks
and a characterisitc of a graph, we can measure how similar the two graphs
are with respect to that measure. In this sense, we will define a
characterisitic of a graph to be some quantifiable metric of the
network structure. An example of such a characteristic would be the
weighted network efficiency of a network \cite{ref1}. Furthermore,
if we have a sample set of real networks that are the
result of the angiogenic process, then we have a way to evaluate how
similar the computational models are to these real samples with respect
to a characterisitic. Hence, the primary focus of the thesis is to determine
and present various characteristics that are the most prevalent to determining
the behaviour of branching as networks that are a result of the angiogenic
process. With such a result, we would be able to provide a process that
will allow the comparision of two networks and how similar they are,
as well as, providing a way to evaluate if current computational models
of angiogenesis are generating networks that are similar to these
characteristics, \cite{ref2}.

\section{Description of the Task}

We can breakdown the task into smaller and concrete components:

\begin{itemize}
    \item[1.] Construct an image analysis pipeline to convert simulated
        and real images of vascular networks into a graph
        representation. This will be done as part of the
        preliminary work to the thesis.
   
    \item[2.] Given that we want to `evaluate' these networks it is important
        to have a sample set of graphs and images of vascular networks that
        are and are not a result of the angiogenic process. As such, we are
        required to collect a sample of images that we can use.

    \item[3.] We then start by evaluating the weighted-network
        efficiency of graphs. To evaluate such a characteristic we want to
        provide both a mathematical and statistical reasoning for
        the strengths and weaknesses of its ability to discern between
        networks that do or do not exhibit the branching tendencies
        as described before.

    \item[4.] Based on the results and the reasonings of this evaluation,
        we are then able to consider and explore other potential combinations
        of characteristics. Thus, the thesis can follow this iterative
        workflow of trying and testing new characteristics to create a
        combination or set of them.

    \item[5.] Throughout this process, we will develop code that allows the
        user to input two images of vascular networks and compare them with
        respect to a particular characteristic.

    \item[6.] With the set of characteristics, we then can analyze the
        various models used to generate such networks and if they are
        `realistic'.

\end{itemize}

Given the iterative nature of the development process we are able to do
these tasks a flexible number of times depending on the time required to
do each iteration. Furthermore, the various mathematical and statisical
reasoning that is used will accumulate as various techniques will be
required and allow us to further evaluate the previous characterisitcs
with the accumulated knowledge. With this in mind, we will want to set the
definitive goal of documenting the findings of 3 different characteristics
to allow for a set pacing when working on the thesis.

\section{Methods}
The thesis will be conducted as a combination of empirical reasearch, as well
as, literature review. Empirically, the task will be to evaluate and
find various characteristics that describe the branching behaviour of
a network. Literature review will be a neccesity with regards
to the graph theory aspects of formulating and evaluating these
quantifications. This research will be a guiding factor when defining exploring
various options.

\section{Relevant Courses}

\begin{enumerate}
    \item \textbf{Introduction to Combinatorics}

    Introductory concepts of graph theory that will allow for a lower barrier
    of entry into graph theory literature

    \item \textbf{Computational Modelling of Cellular Systems}

    Overview of a wide range of mathematical modelling concepts within
    biology to provide many perspectives of the problem

    \item \textbf{Computational Neuroscience \& Image Analysis}

    Experience in developing image analysis pipelines, in particular projects
    were used in a scientific setting

    \item \textbf{Modelling of Complex Systems}

    General knowledge in the modelling of various systems and gained practical
    experience in developing modles

\end{enumerate}

\section{Delimitations}

We would like to primarily focus on developing a good process 
of evaluting how a characteristics can encapsulate the branching behaviour
of a network. As such, the steps of creating an image analysis pipeline and the
usability of the code are areas that could be compressed or expanded on the
basis of available time. For example, tasks such as creating a visualization of a gprah
representation and/or usability of the end code are tasks that can be
done with this in mind. If a loop in the iterative process takes more/less
time than expected we can also change the project to consider more/less
characteristics. However, this wouldn't be ideal.

\section{Time Plan}

We consider the time plan to be done full-time over the fall term of 2023
(week 35-52). However, I will 'begin' during the summer leading up to the thesis
period with preliminary self-study within graph theory and angiogenesis.
Additionally, the implementation of the image analysis portion will be done
in advance.

\begin{itemize}
    \item \textbf{Week 35-36}

        Before we want to evaluate or find possible characterisitcs, then it is
        important to conduct some readings on the computational models
        that are used to generate our networks to gain an understanding of the
        mathematical and statistical reasoning behind their methods. We also
        want to do readings within statistical analysis of networks and
        graph theory in general to get a broader scope of the types of
        measurements or techniques of quantifying behaviours in graphs.
        Collecting sample images of real and generated networks.

    \item \textbf{Week 37-38}

        Continue the readings from before. Begin to evaluate the
        weighted-network efficiency by comparing the difference of
        measurements between graphs of the different behaviours.
        Formulate arguments for why it is and isn't a good measure of
        the branching behaviour. Important to motivate and record
        how we evaluated the weighted-network efficiency characteristics
        to extend and use in later weeks.

    \item \textbf{Week 39-41}

        From the readings we will have the knowledge and can try out
        various characteristics. We want to do the same evaluation process
        as week 37-38 for 1 new characteristic.

    \item \textbf{Week 41-44}

        Begin work on writing draft thesis, laying out the thoughts and methods
        behind the methods used to evaluate our characteristics so far.
        We also want to repeat the process of finding and determining a new
        charactersitic.

    \item \textbf{Week 45-46}

        We now have 3 characteristics that have the reasoning as to why they
        are or aren't a good measure of the branching behaviour. We want to
        formulate the specific reasonings into our draft thesis. Use these
        charactersitics to investigate how similar the generated computational
        models are to the real networks. Document the findings into the draft
        thesis.

    \item \textbf{Week 47-50}

        Finalize the thesis report and accompanying code, as well as prepare
        for presentation and we will allow a buffer period here.

\end{itemize}

\begin{thebibliography}{9}
\bibitem{ref1}
Vilanova, G., Colominas, I., Gomex, H. \emph{Computational Modeling of Tumor-
Induced Angiogenesis.} Archives of Computational Methods in Engineering, 24(4),
1071-1102, 2017. doi:\texttt{https://doi.org/10.1007/s11831-016-9199-7}.

\bibitem{ref2}
Vilanova, G., Colominas, I., Gomex, H. \emph{A mathematical model of angiogenesis:
growth, regression and regrowth}. J. R. Soc. Interface. 14: 20160918. doi:
\texttt{https://doi.org/10.1098/rsif.2016.0918}

\end{thebibliography}

\end{document}
